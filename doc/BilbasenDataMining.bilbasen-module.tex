%
% API Documentation for API Documentation
% Module BilbasenDataMining.bilbasen
%
% Generated by epydoc 3.0.1
% [Thu Dec  4 13:59:41 2014]
%

%%%%%%%%%%%%%%%%%%%%%%%%%%%%%%%%%%%%%%%%%%%%%%%%%%%%%%%%%%%%%%%%%%%%%%%%%%%
%%                          Module Description                           %%
%%%%%%%%%%%%%%%%%%%%%%%%%%%%%%%%%%%%%%%%%%%%%%%%%%%%%%%%%%%%%%%%%%%%%%%%%%%

    \index{BilbasenDataMining \textit{(package)}!BilbasenDataMining.bilbasen \textit{(module)}|(}
\section{Module BilbasenDataMining.bilbasen}

    \label{BilbasenDataMining:bilbasen}
Bilbasen module

This module handles all communication with bilbasen.dk, and it contains 
functions to communicate with bilbasen.dk as well as functions to download 
data from bilbasen.dk. The main part of the module runs the method 
download\_data\_to\_database which is the most important method of the 
module, as this is where bilbasen.dk is crawled and the data is parsed and 
saved in a database.


%%%%%%%%%%%%%%%%%%%%%%%%%%%%%%%%%%%%%%%%%%%%%%%%%%%%%%%%%%%%%%%%%%%%%%%%%%%
%%                               Functions                               %%
%%%%%%%%%%%%%%%%%%%%%%%%%%%%%%%%%%%%%%%%%%%%%%%%%%%%%%%%%%%%%%%%%%%%%%%%%%%

  \subsection{Functions}

    \label{BilbasenDataMining:bilbasen:connect}
    \index{BilbasenDataMining \textit{(package)}!BilbasenDataMining.bilbasen \textit{(module)}!BilbasenDataMining.bilbasen.connect \textit{(function)}}

    \vspace{0.5ex}

\hspace{.8\funcindent}\begin{boxedminipage}{\funcwidth}

    \raggedright \textbf{connect}()

    \vspace{-1.5ex}

    \rule{\textwidth}{0.5\fboxrule}
\setlength{\parskip}{2ex}
    This method returns a httplib connection to bilbasen.dk

\setlength{\parskip}{1ex}
    \end{boxedminipage}

    \label{BilbasenDataMining:bilbasen:extract_car_info}
    \index{BilbasenDataMining \textit{(package)}!BilbasenDataMining.bilbasen \textit{(module)}!BilbasenDataMining.bilbasen.extract\_car\_info \textit{(function)}}

    \vspace{0.5ex}

\hspace{.8\funcindent}\begin{boxedminipage}{\funcwidth}

    \raggedright \textbf{extract\_car\_info}(\textit{listing\_type}, \textit{listing})

    \vspace{-1.5ex}

    \rule{\textwidth}{0.5\fboxrule}
\setlength{\parskip}{2ex}
    This method is used by the method download\_data\_to\_database() and 
    extracts data from a html listing. The data extracted is all the 
    attributes on a car given on a search result site, which is: model, 
    link to detailed page, description, how many km it has done, which year
    it is from, how many horsepowers it has, how many km per liter it does,
    how fast it goes from 0-100 km/t, what the monthly cost is, if it can 
    pull a trailer, and a location. The method saves this data in the 
    namedtupe Car and returns that tuple.

\setlength{\parskip}{1ex}
    \end{boxedminipage}

    \label{BilbasenDataMining:bilbasen:get_date}
    \index{BilbasenDataMining \textit{(package)}!BilbasenDataMining.bilbasen \textit{(module)}!BilbasenDataMining.bilbasen.get\_date \textit{(function)}}

    \vspace{0.5ex}

\hspace{.8\funcindent}\begin{boxedminipage}{\funcwidth}

    \raggedright \textbf{get\_date}()

    \vspace{-1.5ex}

    \rule{\textwidth}{0.5\fboxrule}
\setlength{\parskip}{2ex}
    This method returns the current date

\setlength{\parskip}{1ex}
    \end{boxedminipage}

    \label{BilbasenDataMining:bilbasen:get_car_image_src}
    \index{BilbasenDataMining \textit{(package)}!BilbasenDataMining.bilbasen \textit{(module)}!BilbasenDataMining.bilbasen.get\_car\_image\_src \textit{(function)}}

    \vspace{0.5ex}

\hspace{.8\funcindent}\begin{boxedminipage}{\funcwidth}

    \raggedright \textbf{get\_car\_image\_src}(\textit{link})

    \vspace{-1.5ex}

    \rule{\textwidth}{0.5\fboxrule}
\setlength{\parskip}{2ex}
    Given a link to a car description page, this method returns a link to 
    an image of the car.

\setlength{\parskip}{1ex}
    \end{boxedminipage}

    \label{BilbasenDataMining:bilbasen:create_car_brand_table}
    \index{BilbasenDataMining \textit{(package)}!BilbasenDataMining.bilbasen \textit{(module)}!BilbasenDataMining.bilbasen.create\_car\_brand\_table \textit{(function)}}

    \vspace{0.5ex}

\hspace{.8\funcindent}\begin{boxedminipage}{\funcwidth}

    \raggedright \textbf{create\_car\_brand\_table}(\textit{conn})

    \vspace{-1.5ex}

    \rule{\textwidth}{0.5\fboxrule}
\setlength{\parskip}{2ex}
    This method crawls bilbasen.dk for a list of car brands, and creates 
    and stores these brands in a table called 'Brands' in the database. The
    method is called each time the method download\_data\_to\_database is 
    called.

\setlength{\parskip}{1ex}
    \end{boxedminipage}

    \label{BilbasenDataMining:bilbasen:insert_car_to_table}
    \index{BilbasenDataMining \textit{(package)}!BilbasenDataMining.bilbasen \textit{(module)}!BilbasenDataMining.bilbasen.insert\_car\_to\_table \textit{(function)}}

    \vspace{0.5ex}

\hspace{.8\funcindent}\begin{boxedminipage}{\funcwidth}

    \raggedright \textbf{insert\_car\_to\_table}(\textit{car}, \textit{tablename}, \textit{cursor})

    \vspace{-1.5ex}

    \rule{\textwidth}{0.5\fboxrule}
\setlength{\parskip}{2ex}
    This method takes as input a namedtuple Car, a name of a table in the 
    database and a cursor to the database, and inserts the car into the 
    given table.

\setlength{\parskip}{1ex}
    \end{boxedminipage}

    \label{BilbasenDataMining:bilbasen:download_data_to_database}
    \index{BilbasenDataMining \textit{(package)}!BilbasenDataMining.bilbasen \textit{(module)}!BilbasenDataMining.bilbasen.download\_data\_to\_database \textit{(function)}}

    \vspace{0.5ex}

\hspace{.8\funcindent}\begin{boxedminipage}{\funcwidth}

    \raggedright \textbf{download\_data\_to\_database}(\textit{limit}={\tt None})

    \vspace{-1.5ex}

    \rule{\textwidth}{0.5\fboxrule}
\setlength{\parskip}{2ex}
    This method is the most important one of this module, as this is the 
    one crawling bilbasen.dk to extract data of cars currently on sale. The
    method creates a new table each time it is called, and the table will 
    be named 'AllBrandsDDMMYY', where DDMMYY is the current date. If the 
    table already exist, it will be deleted. The method takes an optional 
    parameter, defining how many pages to crawl. If no parameter is given, 
    it crawls all pages (i.e. all cars) on bilbasen.dk

\setlength{\parskip}{1ex}
    \end{boxedminipage}

    \label{BilbasenDataMining:bilbasen:main}
    \index{BilbasenDataMining \textit{(package)}!BilbasenDataMining.bilbasen \textit{(module)}!BilbasenDataMining.bilbasen.main \textit{(function)}}

    \vspace{0.5ex}

\hspace{.8\funcindent}\begin{boxedminipage}{\funcwidth}

    \raggedright \textbf{main}()

    \vspace{-1.5ex}

    \rule{\textwidth}{0.5\fboxrule}
\setlength{\parskip}{2ex}
    Main method which will run by a command 'python bilbasen.py arg', where
    arg is the number of pages to crawl. Do not provide any argument if all
    pages should be crawled.

\setlength{\parskip}{1ex}
    \end{boxedminipage}


%%%%%%%%%%%%%%%%%%%%%%%%%%%%%%%%%%%%%%%%%%%%%%%%%%%%%%%%%%%%%%%%%%%%%%%%%%%
%%                               Variables                               %%
%%%%%%%%%%%%%%%%%%%%%%%%%%%%%%%%%%%%%%%%%%%%%%%%%%%%%%%%%%%%%%%%%%%%%%%%%%%

  \subsection{Variables}

    \vspace{-1cm}
\hspace{\varindent}\begin{longtable}{|p{\varnamewidth}|p{\vardescrwidth}|l}
\cline{1-2}
\cline{1-2} \centering \textbf{Name} & \centering \textbf{Description}& \\
\cline{1-2}
\endhead\cline{1-2}\multicolumn{3}{r}{\small\textit{continued on next page}}\\\endfoot\cline{1-2}
\endlastfoot\raggedright \_\-\_\-p\-a\-c\-k\-a\-g\-e\-\_\-\_\- & \raggedright \textbf{Value:} 
{\tt \texttt{'}\texttt{BilbasenDataMining}\texttt{'}}&\\
\cline{1-2}
\end{longtable}


%%%%%%%%%%%%%%%%%%%%%%%%%%%%%%%%%%%%%%%%%%%%%%%%%%%%%%%%%%%%%%%%%%%%%%%%%%%
%%                           Class Description                           %%
%%%%%%%%%%%%%%%%%%%%%%%%%%%%%%%%%%%%%%%%%%%%%%%%%%%%%%%%%%%%%%%%%%%%%%%%%%%

    \index{BilbasenDataMining \textit{(package)}!BilbasenDataMining.bilbasen \textit{(module)}!BilbasenDataMining.bilbasen.Car \textit{(class)}|(}
\subsection{Class Car}

    \label{BilbasenDataMining:bilbasen:Car}
\begin{tabular}{cccccccc}
% Line for object, linespec=[False, False]
\multicolumn{2}{r}{\settowidth{\BCL}{object}\multirow{2}{\BCL}{object}}
&&
&&
  \\\cline{3-3}
  &&\multicolumn{1}{c|}{}
&&
&&
  \\
% Line for tuple, linespec=[False]
\multicolumn{4}{r}{\settowidth{\BCL}{tuple}\multirow{2}{\BCL}{tuple}}
&&
  \\\cline{5-5}
  &&&&\multicolumn{1}{c|}{}
&&
  \\
&&&&\multicolumn{2}{l}{\textbf{BilbasenDataMining.bilbasen.Car}}
\end{tabular}

Car(model, link, description, kms, year, hk, kml, kmt, moth, trailer, 
location, price)


%%%%%%%%%%%%%%%%%%%%%%%%%%%%%%%%%%%%%%%%%%%%%%%%%%%%%%%%%%%%%%%%%%%%%%%%%%%
%%                                Methods                                %%
%%%%%%%%%%%%%%%%%%%%%%%%%%%%%%%%%%%%%%%%%%%%%%%%%%%%%%%%%%%%%%%%%%%%%%%%%%%

  \subsubsection{Methods}

    \vspace{0.5ex}

\hspace{.8\funcindent}\begin{boxedminipage}{\funcwidth}

    \raggedright \textbf{\_\_getnewargs\_\_}(\textit{self})

    \vspace{-1.5ex}

    \rule{\textwidth}{0.5\fboxrule}
\setlength{\parskip}{2ex}
    Return self as a plain tuple.  Used by copy and pickle.

\setlength{\parskip}{1ex}
      Overrides: tuple.\_\_getnewargs\_\_

    \end{boxedminipage}

    \label{BilbasenDataMining:bilbasen:Car:__getstate__}
    \index{BilbasenDataMining \textit{(package)}!BilbasenDataMining.bilbasen \textit{(module)}!BilbasenDataMining.bilbasen.Car \textit{(class)}!BilbasenDataMining.bilbasen.Car.\_\_getstate\_\_ \textit{(method)}}

    \vspace{0.5ex}

\hspace{.8\funcindent}\begin{boxedminipage}{\funcwidth}

    \raggedright \textbf{\_\_getstate\_\_}(\textit{self})

    \vspace{-1.5ex}

    \rule{\textwidth}{0.5\fboxrule}
\setlength{\parskip}{2ex}
    Exclude the OrderedDict from pickling

\setlength{\parskip}{1ex}
    \end{boxedminipage}

    \vspace{0.5ex}

\hspace{.8\funcindent}\begin{boxedminipage}{\funcwidth}

    \raggedright \textbf{\_\_new\_\_}(\textit{\_cls}, \textit{model}, \textit{link}, \textit{description}, \textit{kms}, \textit{year}, \textit{hk}, \textit{kml}, \textit{kmt}, \textit{moth}, \textit{trailer}, \textit{location}, \textit{price})

    \vspace{-1.5ex}

    \rule{\textwidth}{0.5\fboxrule}
\setlength{\parskip}{2ex}
    Create new instance of Car(model, link, description, kms, year, hk, 
    kml, kmt, moth, trailer, location, price)

\setlength{\parskip}{1ex}
      \textbf{Return Value}
    \vspace{-1ex}

      \begin{quote}
      a new object with type S, a subtype of T

      \end{quote}

      Overrides: object.\_\_new\_\_

    \end{boxedminipage}

    \vspace{0.5ex}

\hspace{.8\funcindent}\begin{boxedminipage}{\funcwidth}

    \raggedright \textbf{\_\_repr\_\_}(\textit{self})

    \vspace{-1.5ex}

    \rule{\textwidth}{0.5\fboxrule}
\setlength{\parskip}{2ex}
    Return a nicely formatted representation string

\setlength{\parskip}{1ex}
      Overrides: object.\_\_repr\_\_

    \end{boxedminipage}


\large{\textbf{\textit{Inherited from tuple}}}

\begin{quote}
\_\_add\_\_(), \_\_contains\_\_(), \_\_eq\_\_(), \_\_ge\_\_(), \_\_getattribute\_\_(), \_\_getitem\_\_(), \_\_getslice\_\_(), \_\_gt\_\_(), \_\_hash\_\_(), \_\_iter\_\_(), \_\_le\_\_(), \_\_len\_\_(), \_\_lt\_\_(), \_\_mul\_\_(), \_\_ne\_\_(), \_\_rmul\_\_(), \_\_sizeof\_\_(), count(), index()
\end{quote}

\large{\textbf{\textit{Inherited from object}}}

\begin{quote}
\_\_delattr\_\_(), \_\_format\_\_(), \_\_init\_\_(), \_\_reduce\_\_(), \_\_reduce\_ex\_\_(), \_\_setattr\_\_(), \_\_str\_\_(), \_\_subclasshook\_\_()
\end{quote}

%%%%%%%%%%%%%%%%%%%%%%%%%%%%%%%%%%%%%%%%%%%%%%%%%%%%%%%%%%%%%%%%%%%%%%%%%%%
%%                              Properties                               %%
%%%%%%%%%%%%%%%%%%%%%%%%%%%%%%%%%%%%%%%%%%%%%%%%%%%%%%%%%%%%%%%%%%%%%%%%%%%

  \subsubsection{Properties}

    \vspace{-1cm}
\hspace{\varindent}\begin{longtable}{|p{\varnamewidth}|p{\vardescrwidth}|l}
\cline{1-2}
\cline{1-2} \centering \textbf{Name} & \centering \textbf{Description}& \\
\cline{1-2}
\endhead\cline{1-2}\multicolumn{3}{r}{\small\textit{continued on next page}}\\\endfoot\cline{1-2}
\endlastfoot\raggedright d\-e\-s\-c\-r\-i\-p\-t\-i\-o\-n\- & \raggedright Alias for field number 2&\\
\cline{1-2}
\raggedright h\-k\- & \raggedright Alias for field number 5&\\
\cline{1-2}
\raggedright k\-m\-l\- & \raggedright Alias for field number 6&\\
\cline{1-2}
\raggedright k\-m\-s\- & \raggedright Alias for field number 3&\\
\cline{1-2}
\raggedright k\-m\-t\- & \raggedright Alias for field number 7&\\
\cline{1-2}
\raggedright l\-i\-n\-k\- & \raggedright Alias for field number 1&\\
\cline{1-2}
\raggedright l\-o\-c\-a\-t\-i\-o\-n\- & \raggedright Alias for field number 10&\\
\cline{1-2}
\raggedright m\-o\-d\-e\-l\- & \raggedright Alias for field number 0&\\
\cline{1-2}
\raggedright m\-o\-t\-h\- & \raggedright Alias for field number 8&\\
\cline{1-2}
\raggedright p\-r\-i\-c\-e\- & \raggedright Alias for field number 11&\\
\cline{1-2}
\raggedright t\-r\-a\-i\-l\-e\-r\- & \raggedright Alias for field number 9&\\
\cline{1-2}
\raggedright y\-e\-a\-r\- & \raggedright Alias for field number 4&\\
\cline{1-2}
\multicolumn{2}{|l|}{\textit{Inherited from object}}\\
\multicolumn{2}{|p{\varwidth}|}{\raggedright \_\_class\_\_}\\
\cline{1-2}
\end{longtable}

    \index{BilbasenDataMining \textit{(package)}!BilbasenDataMining.bilbasen \textit{(module)}!BilbasenDataMining.bilbasen.Car \textit{(class)}|)}
    \index{BilbasenDataMining \textit{(package)}!BilbasenDataMining.bilbasen \textit{(module)}|)}
